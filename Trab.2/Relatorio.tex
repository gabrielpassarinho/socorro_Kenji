\documentclass{article}
\usepackage[utf8]{inputenc}
\usepackage{amsmath}
\usepackage{verbatim}
\usepackage{amsfonts}
\usepackage{graphicx}
\usepackage{listings}
%
\title{%
    2º Projeto da disciplina Estruturas de Dados II \\
     \large Análise assintótica de algoritmos de ordenação}
\author{Danillo Mendes Santiago 10414592\\Gabriel Passarelli 11218480\\ Marcelo Kenji Noda 11275359}
%
\begin{document}
%
\maketitle
%
\section{Introdução}
O propósito deste texto é analisar a complexidade de tempo de algoritmos de busca sequancial e de busca por espalhamento. Eles foram implementados na parte prática do projeto, em linguagem C de programação, e seguindo os templates fornecidos na proposta do trabalho.\par
%
O texto está dividido em seções, e há uma seção para cada algoritmo implementado. Em cada um delas, apresentamos os resultados das medições do tempo de execução de cada algoritmo, e comentamos detalher pertinentes de nossa implementação. Nessa parte, incluímos tabelas para facilitar a visualização dos dados de tempo. Como cada medição foi realizada três vezes, as tabelas contêm também medidas de disperção dos dados (o desvio-padrão). Por fim, há uma seção apresentando as conclusões desenhadas pelo grupo.
%
\section{Análise dos algoritmos de busca sequancial}
\subsection{Busca sequencial simples}
%
%
%
\subsection{Busca sequencial com método \textit{mover-para-frente}}
%
%
%
\subsection{Busca sequencial com método de \textit{transposição}}
%
%
%
\subsection{Busca sequencial com índice primário}
%
%
%
\section{Análise dos algoritmos de busca por espalhamento}
\subsection{Hash com \textit{overflow progressivo}}
Através da tabela 1
\begin{table}[!]
    \begin{tabular}{c|c|c|c}
         & Inserção & Busca & Colisões \\ 
        \hline
        h\_div & $\mu = 1.32\cdot 10^{-1},\;\sigma = 4.9\cdot10^{-4}$ & $\mu=3.29\cdot 10^{-1},\;\sigma = 3.36\cdot 10^{-3}$ & 28558 \\
        \hline
        h\_mul & $\mu=4.08\cdot10^{-1},\;\sigma=4.25\cdot 10^{-3}$ & $\mu=9.4\cdot 10^{-1},\;\sigma=3.33\cdot 10^{-3}$ & 34333\\
        \hline
    \end{tabular}
    \caption{Medidas de tempo e do número de colisões em cada hash. $\mu$ indica a média das medidas e $\sigma$, o desvio padrão. Todas as medidas estão em segundos.}
\end{table}\par
jnkjnk
%
\subsection{Hash com \textit{hash duplo}}
pique pique 
\begin{table}[!]
    \begin{tabular}{c|c|c|c}
         & Inserção & Busca & Colisões \\ 
        \hline
        h & $\mu = 1.90\cdot 10^{-1},\;\sigma = 8.30\cdot10^{-3}$ & $\mu=3.89\cdot 10^{-1},\;\sigma = 3.44\cdot 10^{-3}$ & 33960 \\
    \end{tabular}
    \caption{Medidas de tempo e do número de colisões no hash duplo. $\mu$ indica a média das medidas e $\sigma$, o desvio padrão. Todas as medidas estão em segundos.}
\end{table}\par
%
\subsection{Hash aberto com lista encadeada não ordenada}
%
\begin{table}[!]
    \begin{tabular}{c|c|c|c}
         & Inserção & Busca & Colisões \\ 
        \hline
        h\_div & $\mu = 3.88\cdot 10^{-2},\;\sigma = 7.0\cdot10^{-4}$ & $\mu=6.03\cdot 10^{-2},\;\sigma = 6.3\cdot 10^{-4}$ & 28558 \\
        \hline
        h\_mul & $\mu=1.26\cdot10^{-1},\;\sigma=2.04\cdot 10^{-3}$ & $\mu=2.43\cdot 10^{-1},\;\sigma=4.33\cdot 10^{-3}$ & 34333\\
        \hline
    \end{tabular}
    \caption{Medidas de tempo e do número de colisões em cada hash. $\mu$ indica a média das medidas e $\sigma$, o desvio padrão. Todas as medidas estão em segundos.}
\end{table}\par
%
\section{Conclusão}
%% incluir grafico %%
%
%
%
\end{document}
